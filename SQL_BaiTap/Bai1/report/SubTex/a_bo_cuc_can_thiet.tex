\iffalse

Phần mở đầu

1.1- Tên đề tài
1.2- Lý do chọn đề tài (tính cấp thiết của vấn đề)
1.3- Mục đích nghiên cứu
1.4- Khách thể và đối tượng nghiên cứu
1.5- Giả thuyết khoa học
1.6- Nhiệm vụ nghiên cứu
1.7- Phạm vi nghiên cứu
1.8- Những luận điểm báo cáo kết quả
1.9- Đóng góp mới của đề tài
1.10- Cơ sở phương pháp luận và phương pháp nghiên cứu


Nội dung
Cơ sở lý luận và thực tiễn của vấn đề
Nội dung và kết quả nghiên cứu
Những giải pháp và khuyến nghị



Kết luận
Tổng hợp các kết quả nghiên cứu
Nêu rõ vấn đề nào đã được giải quyết và chưa được giả quyết
Vấn đề mới nảy sinh cần tiếp tục nghiên cứu
Kết luận cần được trình bày súc tích, cô đọng, sâu sắc, ngắn gọn
Không có lời bàn và bình luận gì thêm


Tài liệu tham khảo



Phụ lục
Phụ lục, các câu hỏi điều tra, các bài tập trắc nghiệm, 
bảng hướng dẫn, chỉ dẫn hoặc ước chú, các biểu bảng, 
số liệu, hình vẽ, biểu đồ, đồ thị, 
phần giải thích thuật ngữ, phần tra cứu theo đề mục hay tác giả, 
các công trình (bài viết) đi sâu từng khía cạnh của đề tài (nếu có).

\fi