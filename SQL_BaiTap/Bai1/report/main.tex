\documentclass[12pt,a4paper,titlepage]{article}
\usepackage[utf8]{inputenc}
\usepackage{amsmath}
\usepackage{amsfonts}
\usepackage{amssymb}
\usepackage{amsthm}    % môi trường math
\usepackage{vntex}     % tiếng việt
\usepackage{graphicx}  % chèn hình ảnh
\usepackage{xcolor}    % màu sắc 
\usepackage{qtree}     % vẽ sơ đồ cây
\usepackage{url}
\usepackage{framed}    % đóng khung đoạn văn
\usepackage{verbatim}  % định dạng text
\usepackage{lastpage}  % trang ... / (trang cuối tài liệu)
\usepackage{diagbox}   % kẻ đường chéo trong bảng
\usepackage[left=2.5cm,right=2.5cm,top=2.5cm,bottom=2.5cm]{geometry}  % căn lề
\usepackage{sidecap}

\usepackage{fancyhdr} 
\usepackage{lastpage}
\pagestyle{fancy} 
\lhead{
\begin{tabular}{rl}
	\begin{picture}(25,15)(0,0)
		\put(0,-8){\includegraphics[width=8mm, height=8mm]{images/LogoBK.jpg}}
	\end{picture}
	&
	\begin{tabular}{l}
		\textbf{Trường Đại học Bách Khoa}\\
		\textbf{Khoa Khoa Học và Kỹ Thuật Máy Tính}
	\end{tabular} 	
\end{tabular}
}
\chead{}
\rhead{} 
\lfoot{\textbf{{Hệ cơ sở dữ liệu - Bài tập lớn}}}
\cfoot{}
\rfoot{\textbf{Trang {\thepage}/{\pageref{LastPage}}}}
\renewcommand{\headrulewidth}{0.2pt}
\renewcommand{\footrulewidth}{0.2pt}
% frame cho text
\usepackage[framemethod=default]{mdframed} 
\mdfdefinestyle{green}{
	innerleftmargin=20,innerrightmargin=20,
	innertopmargin = 15, innerbottommargin = 15,
	hidealllines=true, 
	backgroundcolor=green!30, 
}
\mdfdefinestyle{pink}{
	innerleftmargin=20,innerrightmargin=20,
	innertopmargin = 15, innerbottommargin = 15,
	hidealllines=true,
	backgroundcolor=red!10
}
\mdfdefinestyle{blue1}{
	innerleftmargin=20,innerrightmargin=20,
	innertopmargin = 15, innerbottommargin = 15,
	hidealllines=true, 
	backgroundcolor=blue!10
}
\mdfdefinestyle{blue2}{
	innerleftmargin=50,
	innertopmargin = 15, innerbottommargin = 15,
	hidealllines=true, 
	backgroundcolor=blue!5
}

% draw the line
\usepackage{tikz}
\usetikzlibrary{calc}
\newcommand\HRule{\rule{\textwidth}{1pt}}

% link color
\usepackage{hyperref}
\hypersetup{
	colorlinks,                   % surround the links by color frames
	linkcolor = {red!50!black},   % color of internal links (sections, pages, etc.)
	citecolor = {blue!60!black},  % color of citation links (bibliography)
	urlcolor  = {blue!60!black},  % color of URL links (mail, web)
	filecolor = {blue!60!black},  % color of file links
}

% other

\usepackage[font={small,it}]{caption}
\usepackage[linguistics]{forest}



\author{Phạm Minh Hiếu - 1611046@hcmut.edu.vn}
\title{Báo cáo BTL 01 OS}

\begin{document}
	
	\begin{titlepage}
	
\begin{tikzpicture}[remember picture, overlay]
\draw[line width = 3pt] ($(current page.north west) + (2cm,-2cm)$) rectangle ($(current page.south east) + (-2cm,2cm)$);
\end{tikzpicture}
	
\begin{center}  
\textbf{ĐẠI HỌC QUỐC GIA THÀNH PHỐ HỒ CHÍ MINH\\TRƯỜNG ĐẠI HỌC BÁCH KHOA\\KHOA KHOA HỌC VÀ KĨ THUẬT MÁY TÍNH}

\vspace{1cm}

\includegraphics[width=3.5cm,height=3.5cm]{images/LogoBK}

\vspace{1cm}

{\LARGE \textbf{\textcolor{red}{HỆ CƠ SỞ DỮ LIỆU}}}
\\

\vspace{0.5cm}

\begin{table*}[h!]
	\centering
	\begin{tikzpicture}\draw (-9.5,0) -- (4,0);\end{tikzpicture}
{\large \textbf{\textcolor{black}{Assignment}}}\\
\vspace{0.5cm}
{\LARGE \textbf{\textcolor{blue}{Phát triển ứng dụng với cơ sở dữ liệu}}}
	\begin{tikzpicture}\draw (-9.5,0) -- (4,0);\end{tikzpicture}
\end{table*}

\vspace{0.5cm}

\begin{tabular}{ll}
		\multicolumn{1}{r}{\textbf{GVHD:}} 
		& \textbf{ThS. Trương Quỳnh Chi}\\
		& \\
		
		\multicolumn{1}{r}{\textbf{Nhóm:}} 
		& \textbf{L07_THA}\\		
		
		\multicolumn{1}{r}{\textbf{DSSV:}}
		& \textbf{Phạm Hòa - 1711440}\\
		& \textbf{Dương Văn Tài - 1613001}\\
		& \textbf{Nguyễn Công Anh - 1710477}\\
		
\end{tabular}

\vspace{4.0cm}  

{\normalsize \textbf{Thành phố Hồ Chí Minh, \today}}

\end{center}

\end{titlepage}


	\fontsize{12pt}{16} \selectfont
	\newpage\tableofcontents\thispagestyle{empty} 
	\newpage\setcounter{page}{1}
	
	\paragraph*{}\vspace{1.5cm}
	\listoffigures\addcontentsline{toc}{section}{\listfigurename}
	
	\newpage
	%\listoftables %\addcontentsline{toc}{section}{\listtablename}
	
	
	
	%\iffalse

Phần mở đầu

1.1- Tên đề tài
1.2- Lý do chọn đề tài (tính cấp thiết của vấn đề)
1.3- Mục đích nghiên cứu
1.4- Khách thể và đối tượng nghiên cứu
1.5- Giả thuyết khoa học
1.6- Nhiệm vụ nghiên cứu
1.7- Phạm vi nghiên cứu
1.8- Những luận điểm báo cáo kết quả
1.9- Đóng góp mới của đề tài
1.10- Cơ sở phương pháp luận và phương pháp nghiên cứu


Nội dung
Cơ sở lý luận và thực tiễn của vấn đề
Nội dung và kết quả nghiên cứu
Những giải pháp và khuyến nghị



Kết luận
Tổng hợp các kết quả nghiên cứu
Nêu rõ vấn đề nào đã được giải quyết và chưa được giả quyết
Vấn đề mới nảy sinh cần tiếp tục nghiên cứu
Kết luận cần được trình bày súc tích, cô đọng, sâu sắc, ngắn gọn
Không có lời bàn và bình luận gì thêm


Tài liệu tham khảo



Phụ lục
Phụ lục, các câu hỏi điều tra, các bài tập trắc nghiệm, 
bảng hướng dẫn, chỉ dẫn hoặc ước chú, các biểu bảng, 
số liệu, hình vẽ, biểu đồ, đồ thị, 
phần giải thích thuật ngữ, phần tra cứu theo đề mục hay tác giả, 
các công trình (bài viết) đi sâu từng khía cạnh của đề tài (nếu có).

\fi
	
	\section{Giới thiệu bài tập lớn}
	Yêu cầu của bài tập lớn số 1 là thực hiện các bước để bổ sung một system call
	vào trong kernel linux. System call này có công dụng để hiển thị một số thông
	tin về một process với PID được cung cấp. \\
	Báo cáo này sẽ mô tả từng bước để thêm system call "procsched" vào trong kernel
	của Linux.\\
	%\subsection{Lí do chọn đề tài}
	%\subsection{Nội dung đề tài}
	%\section{Cơ sở lí thuyết}
	\section{Mô tả các bước tiến hành}
	\subsection{Thêm một system call mới}
	
	Để tiến hành thêm một system call mới, trước hết ta cần phải thiết lập môi
	trường thực hiện cũng như một số công cụ cần thiết như: thiết lập máy ảo
	(virtual machine), cài đặt các build-essential metapackage và kernel
	package...\\
	
	$\bullet$Tại sao cần phải cài đặt kernel package?\\
	\textbf{Trả lời:} Chúng ta phải cài đặt kernel package để tự động hóa các bước cần thiết khi biên dịch và cài đặt một kernel tùy chỉnh (custom kernel).\\
	
	Bước tiếp theo cần làm là tải về kernel source có sẵn để tùy chỉnh\\
	
	$\bullet$ Tại sao ta phải dùng một mã nguồn kernel khác để compile, ta có thể compile trực tiếp bản kernel đang chạy trên hệ điều hành được không?\\
	\textbf{Trả lời:} Không được. Vì bản kernel đang chạy trên hệ điều hành là bản kernel đã được build ra những file thực thi và cài vào kernel hệ thống. Việc chỉnh sửa trên này là bất khả thi cũng như việc compile trực tiếp bản kernel đó sẽ không thành công.\\ 
	
	Để thêm một system call mới vào kernel, trong thư mục chứa kernel source vừa
	được tải về và giải nén, ta truy cập thư mục theo đường dẫn
	"arch/x86/entry/syscalls", sau đó truy cập và chỉnh sửa hai file syscall\_32.tbl
	và syscall\_64.tbl bằng cách thêm vào cuối mỗi file một dòng mã chứa các thông
	số. Ví dụ đối với file syscall\_32.tbl, thông số number chính là số thứ tự của
	dòng cuối cùng cộng thêm một:\\
	\texttt{ [number]	i386	procsched	sys\_procsched }\\
	
	$\bullet$Ý nghĩa của những phần còn lại? \\
	\textbf{Trả lời:} ie86 chỉ hệ điều hành 32 bit; procsched
	là tên của syscall, sys\_procsched là tên của file mã nguồn implement syscall
	đó.\\
	
	Tương tự với file syscall\_64.tbl, ta cũng thêm dòng sau:\\
	\texttt{[number] x32 procsched sys\_procsched}\\
	
	Công việc tiếp theo là định nghĩa system call này trong file header của kernel.
	Mở file \texttt{include/linux/syscalls.h} và thêm những dòng sau vào cuối file:
	\\
	\texttt{struct proc\_segs;}\\
	\texttt{asmlinkage long sys\_procsched( int pid, struct proc\_segs * info);}\\
	
	$\bullet$Ý nghĩa của mỗi dòng trên?\\
	\textbf{Trả lời:} Dòng đầu tiên ta định nghĩa một struct tên
	là proc\_segs; dòng thứ hai prototype của hàm hiện thực syscall.\\
	
	\subsection{Hiện thực system call "procsched"}
	Để làm việc này, ta tạo một file mã nguồn sys\_procsched.c trong
	thư mục \texttt{arch/x86/kernel} (file mã nguồn có đi kèm với báo cáo) \\
	
	Tiếp theo ta tiến hành chỉnh sửa Makefile bằng cách thêm một dòng mã vào cuối
	file \texttt{obj-y += sys\_procsched.o} \\
	
	\subsection{Biên dịch và cài đặt}
	Để tiến hành biên dịch kernel, từ thư mục ngoài cùng của kernel source
	(~/kernelbuild/linux-4.4.56) ta sử dụng lệnh: \texttt{\$ make} hoặc \texttt{\$
		make -j 4} \\
	
	Sau khi biên dịch thành công, ta dùng lệnh \texttt{\$ make modules} hoặc lệnh
	\texttt{\$ make -j 4 modules}\\
	
	$\bullet$ Ý nghĩa của lệnh \texttt{make} và \texttt{make modules}?\\
	\textbf{Trả lời:}
	lệnh \texttt{make} có nhiệm vụ gọi Makefile để biên dịch và liên kết file kernel
	image và tạo file "vmlinuz", file này sẽ được bootloader load vào bộ nhớ khi
	khởi động. Còn lệnh "make modules" sẽ biên dịch những file riêng lẻ tạo những
	file object rồi liên kết chúng lại tạo kernel đã được build. Kernel đã sẵn sàng
	để được cài đặt.\\
	
	Để cài đặt modules kernel trong thư mục modules, ta sử dụng lệnh \texttt{\$ sudo
		make modules\_install} hoặc \texttt{\$ sudo make -j 4 modules\_install} \\
	
	Sau đó cài đặt kernel đã được build vào trong vmlinuz, ta dùng lệnh \texttt{\$
		sudo make install} hoặc \texttt{\$ sudo make -j 4 install}\\
	
	Để chắc chắn rằng system call được cài đặt thành công sẽ hoạt động, ta viết một
	chương trình C nhỏ có chức năng gọi syscall này.\\
	
	$\bullet$ Tại sao chương trình này có thể chỉ ra liệu syscall mới này hoạt động
	hay không? \\
	\textbf{Trả lời:} Vì chương trình này có sử dụng lời gọi trực tiếp đến syscall ta vừa
	hiên thực. Cụ thể với lệnh: \texttt{syscall([number\_32], 1, info);} sẽ trả về
	giá trị chứng tỏ các thông số của process được truyền vào có được đưa vào struct
	info hay không. Từ đó ta có thể kết luận là syscall có hoạt động hay không.\\
	
	
	\subsection{Tạo API cho system call "procsched"}
	Tạo API (đóng gói syscall) giúp dễ sử dụng cho sau này. Ta lần lượt tạo file
	"procsched.h" và "procsched.c" và hiện thực chúng.\\
	
	$\bullet$ Tại sao ta phải định nghĩa lại proc\_segs struct trong khi đã định
	nghĩa nó ở trong kernel? \\
	\textbf{Trả lời:} Vì proc\_segs struct ta đặt trong file hiện thực
	syscall (sys\_procsched.c) và hiện tại ta chưa gọi lệnh syscall và ta cũng chưa
	thể dùng struct này như struct trong thư viện đã có sẵn,\\
	
	Sau khi đã hiện thực hàm đóng gói, ta cần cài đặt nó vào máy ảo. Trước tiên ta
	copy file header procsched.h vào /usr/include để đảm bảo bất cứ ai cũng có quyền
	truy cập hàm này. \\
	
	$\bullet$ Tại sao việc copy file header này lại cần quyền root?\\
	\textbf{Trả lời:} Vì đây là việc can thiệp vào thành phần kernel của hệ thống nên cần phải có quyền root để đảm bảo an toàn cho hệ thống.\\
	
	Tiếp theo ta tiến hành biên dịch file mã nguồn vừa hiện thực như một đối tượng được chia sẻ cho phép người dùng sử dụng system call vừa tạo vào chương trình của họ. Để làm điều đó ta thực hiện lệnh: \texttt{\$ gcc -shared -fpic procsched.c -o libprocsched.so}. Sau khi biên dịch thành công ta copy file kết quả (libprocsched.so) vào trong thư mục /usr/lib với quyền root.\\
	
	$\bullet$ Tại sao ta phải đặt "-share" và "-fpic" vào trong câu lệnh biên dịch?\\
	\textbf{Trả lời:} "-share" và "-fpic" để cho hệ điều hành biết đoạn mã được biên dịch sẽ có thể được chia sẻ giữa các tiến trình.\\
	\section{Kiểm tra và so sánh kết quả}
	Kết quả của hàm kiểm tra syscall và module kernel được thể hiện ở hình bên dưới với pid test bằng 1, kết quả cho thấy syscall hoạt động bình thường sau khi được cài đặt vào trong kernel (các giá trị trả về tương đồng với hàm được test trong module kernel) :
	
	\begin{figure}[h!]
		\centering
		\caption{Kết quả test syscall sau khi đã đóng gói}
		\includegraphics[width=0.5\textwidth]{images/test_syscall.png}
	\end{figure}
	
	
	\begin{figure}[h!]
		\centering
		\caption{Kết quả test module kernel}
		\includegraphics[width=0.5\textwidth]{images/test_module_kernel.png}
	\end{figure}
	%% insert \cite{name1, name2, ...}

\section*{Tài liệu tham khảo}
\renewcommand{\refname}{Tài liệu tham khảo}
\addcontentsline{toc}{section}{\bibname}

\bibliographystyle{acm}
\bibliography{SubTex/bibtex}


\iffalse 

\input{SubTex/bibtex.bib} 

\fi


	%\section*{Phụ lục} 
\addcontentsline{toc}{section}{Phụ lục} \label{phuluc}


	
\end{document}
